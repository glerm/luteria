 %% abtex2-modelo-trabalho-academico.tex, v-1.9.2 laurocesar

%% Copyright 2012-2014 by abnTeX2 group at http://abntex2.googlecode.com/ 
%%
%% This work may be distributed and/or modified under the
%% conditions of the LaTeX Project Public License, either version 1.3
%% of this license or (at your option) any later version.
%% The latest version of this license is in
%%   http://www.latex-project.org/lppl.txt
%% and version 1.3 or later is part of all distributions of LaTeX
%% version 2005/12/01 or later.
%%
%% This work has the LPPL maintenance status `maintained'.
%% 
%% The Current Maintainer of this work is the abnTeX2 team, led
%% by Lauro César Araujo. Further information are available on 
%% http://abntex2.googlecode.com/
%%
%% This work consists of the files abntex2-modelo-trabalho-academico.tex,
%% abntex2-modelo-include-comandos and abntex2-modelo-references.bib
%%

% ------------------------------------------------------------------------
% ------------------------------------------------------------------------
% abnTeX2: Modelo de Trabalho Academico (tese de doutorado, dissertacao de
% mestrado e trabalhos monograficos em geral) em conformidade com 
% ABNT NBR 14724:2011: Informacao e documentacao - Trabalhos academicos -
% Apresentacao
% ------------------------------------------------------------------------
% ------------------------------------------------------------------------

\documentclass[
	% -- opções da classe memoir --
	12pt,				% tamanho da fonte
	openright,			% capítulos começam em pág ímpar (insere página vazia caso preciso)
	twoside,			% para impressão em verso e anverso. Oposto a oneside
	a4paper,			% tamanho do papel. 
	% -- opções da classe abntex2 --
	%chapter=TITLE,		% títulos de capítulos convertidos em letras maiúsculas
	%section=TITLE,		% títulos de seções convertidos em letras maiúsculas
	%subsection=TITLE,	% títulos de subseções convertidos em letras maiúsculas
	%subsubsection=TITLE,% títulos de subsubseções convertidos em letras maiúsculas
	% -- opções do pacote babel --
	english,			% idioma adicional para hifenização
	french,				% idioma adicional para hifenização
	spanish,			% idioma adicional para hifenização
	brazil				% o último idioma é o principal do documento
	]{abntex2}
%https://code.google.com/p/abntex2/wiki/Texmaker
% ---
% Pacotes básicos 
% ---
\usepackage{lmodern}			% Usa a fonte Latin Modern			
\usepackage[T1]{fontenc}		% Selecao de codigos de fonte.
\usepackage[utf8]{inputenc}		% Codificacao do documento (conversão automática dos acentos)
\usepackage{lastpage}			% Usado pela Ficha catalográfica
\usepackage{indentfirst}		% Indenta o primeiro parágrafo de cada seção.
\usepackage{color}				% Controle das cores
\usepackage{graphicx}			% Inclusão de gráficos
\usepackage{microtype} 			% para melhorias de justificação
\usepackage{url}
\usepackage[table,xcdraw]{xcolor}

% ---
		
% ---
% Pacotes adicionais, usados apenas no âmbito do Modelo Canônico do abnteX2
% ---
\usepackage{lipsum}				% para geração de dummy text
% ---

% ---
% Pacotes de citações
% ---
\usepackage[brazilian,hyperpageref]{backref}	 % Paginas com as citações na bibl
\usepackage[alf]{abntex2cite}	% Citações padrão ABNT

%%%%%%%%%%% syntax highlight %%%%%%%%%%%%%%%%%%%%%%%%%%%%%%%%%%%%%%%%%%%%%%%%%%%%%


\usepackage{listings}
\definecolor{maroon}{rgb}{0.5,0,0}
\definecolor{darkgreen}{rgb}{0,0.5,0}
\definecolor{deepblue}{rgb}{0,0,0.5}
\definecolor{deepred}{rgb}{0.6,0,0}
\definecolor{purple}{rgb}{0.5,0,0.5}
\definecolor{deepgreen}{rgb}{0,0.5,0}


 






%%%%%%%%%%%%%%%%%%%%%%%%%%%%%%%%%%%%%%%%%%%%%%%%%%%%%%% 


% --- 
% CONFIGURAÇÕES DE PACOTES
% --- 

% ---
% Configurações do pacote backref
% Usado sem a opção hyperpageref de backref
\renewcommand{\backrefpagesname}{Citado na(s) página(s):~}
% Texto padrão antes do número das páginas
\renewcommand{\backref}{}
% Define os textos da citação
\renewcommand*{\backrefalt}[4]{
	\ifcase #1 %
		Nenhuma citação no texto.%
	\or
		Citado na página #2.%
	\else
		Citado #1 vezes nas páginas #2.%
	\fi}%
% ---

% ---
% Informações de dados para CAPA e FOLHA DE ROSTO
% ---
\titulo{Luteria Composicional de algoritmos pós-tonais }
\autor{Guilherme Rafael Soares}
%\local{Brasil}
\data{10 de julho de 2014, v0.6-Qualificação}
\orientador{Prof. Dr. Daniel Quaranta}
%\coorientador{Equipe \abnTeX}
\instituicao{%
  UFJF - Universidade Federal de Juiz de Fora
  \par
  Instituto de Artes e Design
  \par
  Programa de Pós-Graduação em Artes, Cultura e Linguagens}
\tipotrabalho{Tese (Mestrado)}
% O preambulo deve conter o tipo do trabalho, o objetivo, 
% o nome da instituição e a área de concentração 
\preambulo{Prévia da dissertação para a banca de qualificação para o Mestrado em Arte, Cultura e Linguagens do IAD-UFJF.}
% ---


% ---
% Configurações de aparência do PDF final

% alterando o aspecto da cor azul
\definecolor{blue}{RGB}{41,5,195}

% informações do PDF
\makeatletter
\hypersetup{
     	%pagebackref=true,
		pdftitle={\@title}, 
		pdfauthor={\@author},
    	pdfsubject={\imprimirpreambulo},
	    pdfcreator={LaTeX with abnTeX2},
		pdfkeywords={abnt}{latex}{abntex}{abntex2}{trabalho acadêmico}, 
		colorlinks=true,       		% false: boxed links; true: colored links
    	linkcolor=blue,          	% color of internal links
    	citecolor=blue,        		% color of links to bibliography
    	filecolor=magenta,      		% color of file links
		urlcolor=blue,
		bookmarksdepth=4
}
\makeatother
% --- 

% --- 
% Espaçamentos entre linhas e parágrafos 
% --- 

% O tamanho do parágrafo é dado por:
\setlength{\parindent}{1.3cm}

% Controle do espaçamento entre um parágrafo e outro:
\setlength{\parskip}{0.2cm}  % tente também \onelineskip

% ---
% compila o indice
% ---
\makeindex
% ---

% ----
% Início do documento
% ----
\begin{document}

% Retira espaço extra obsoleto entre as frases.
\frenchspacing 

% ----------------------------------------------------------
% ELEMENTOS PRÉ-TEXTUAIS
% ----------------------------------------------------------
% \pretextual

% ---
% Capa
% ---
\imprimircapa
% ---


% ---
% Folha de rosto
% (o * indica que haverá a ficha bibliográfica)
% ---
\imprimirfolhaderosto*
% ---

% ---
% Inserir a ficha bibliografica
% ---

% Isto é um exemplo de Ficha Catalográfica, ou ``Dados internacionais de
% catalogação-na-publicação''. Você pode utilizar este modelo como referência. 
% Porém, provavelmente a biblioteca da sua universidade lhe fornecerá um PDF
% com a ficha catalográfica definitiva após a defesa do trabalho. Quando estiver
% com o documento, salve-o como PDF no diretório do seu projeto e substitua todo
% o conteúdo de implementação deste arquivo pelo comando abaixo:
%
% \begin{fichacatalografica}
%     \includepdf{fig_ficha_catalografica.pdf}
% \end{fichacatalografica}
\begin{fichacatalografica}
	\vspace*{\fill}					% Posição vertical
	\hrule							% Linha horizontal
	\begin{center}					% Minipage Centralizado
	\begin{minipage}[c]{12.5cm}		% Largura
	
	\imprimirautor
	
	\hspace{0.5cm} \imprimirtitulo  / \imprimirautor. --
	\imprimirlocal, \imprimirdata-
	
	\hspace{0.5cm} \pageref{LastPage} p. : il. (algumas color.) ; 30 cm.\\
	
	\hspace{0.5cm} \imprimirorientadorRotulo~\imprimirorientador\\
	
	\hspace{0.5cm}
	\parbox[t]{\textwidth}{\imprimirtipotrabalho~--~\imprimirinstituicao,
	\imprimirdata.}\\
	
	\hspace{0.5cm}
		1. Palavra-chave1.
		2. Palavra-chave2.
		I. Orientador: Prof. Dr. Daniel Quaranta
		II. UFJF - Universidade Federal de Juiz de Fora.
		III. Instituto de Artes e Design
		IV. \imprimirtitulo \\ 			
	
	\hspace{8.75cm} CDU 02:141:005.7\\
	
	\end{minipage}
	\end{center}
	\hrule
\end{fichacatalografica}
% ---

% ---
% Inserir errata
% ---
%\begin{errata}
%Elemento opcional da \citeonline[4.2.1.2]{NBR14724:2011}. Exemplo:
%
%\vspace{\onelineskip}
%
%FERRIGNO, C. R. A. \textbf{Tratamento de neoplasias ósseas apendiculares com
%reimplantação de enxerto ósseo autólogo autoclavado associado ao plasma
%rico em plaquetas}: estudo crítico na cirurgia de preservação de membro em
%cães. 2011. 128 f. Tese (Livre-Docência) - Faculdade de Medicina Veterinária e
%Zootecnia, Universidade de São Paulo, São Paulo, 2011.
%
%\begin{table}[htb]
%\center
%\footnotesize
%\begin{tabular}{|p{1.4cm}|p{1cm}|p{3cm}|p{3cm}|}
%  \hline
%   \textbf{Folha} & \textbf{Linha}  & \textbf{Onde se lê}  & \textbf{Leia-se}  \\
%    \hline
%    1 & 10 & auto-conclavo & autoconclavo\\
%   \hline
%\end{tabular}
%\end{table}

%\end{errata}
% ---

% ---
% Inserir folha de aprovação
% ---

% Isto é um exemplo de Folha de aprovação, elemento obrigatório da NBR
% 14724/2011 (seção 4.2.1.3). Você pode utilizar este modelo até a aprovação
% do trabalho. Após isso, substitua todo o conteúdo deste arquivo por uma
% imagem da página assinada pela banca com o comando abaixo:
%
% \includepdf{folhadeaprovacao_final.pdf}
%
\begin{folhadeaprovacao}

  \begin{center}
    {\ABNTEXchapterfont\large\imprimirautor}

    \vspace*{\fill}\vspace*{\fill}
    \begin{center}
      \ABNTEXchapterfont\bfseries\Large\imprimirtitulo
    \end{center}
    \vspace*{\fill}
    
    \hspace{.45\textwidth}
    \begin{minipage}{.5\textwidth}
        \imprimirpreambulo
    \end{minipage}%
    \vspace*{\fill}
   \end{center}
        
   Trabalho aprovado \imprimirlocal, 13 de fevereiro de 2015:

   \assinatura{\textbf{\imprimirorientador} \\ Orientador} 
   \assinatura{\textbf{Professor} \\ Convidado 1}
   \assinatura{\textbf{Professor} \\ Convidado 2}
   %\assinatura{\textbf{Professor} \\ Convidado 3}
   %\assinatura{\textbf{Professor} \\ Convidado 4}
      
   \begin{center}
    \vspace*{0.5cm}
    {\large\imprimirlocal}
    \par
    {\large\imprimirdata}
    \vspace*{1cm}
  \end{center}
  
\end{folhadeaprovacao}
% ---

% ---
% Dedicatória
% ---
%\begin{dedicatoria}
%   \vspace*{\fill}
%   \centering
%   \noindent
%   \textit{ Este trabalho é dedicado às crianças adultas que,\\
%   quando pequenas, sonharam em se tornar cientistas.} \vspace*{\fill}
%\end{dedicatoria}
% ---

% ---
% Agradecimentos
% ---
%\begin{agradecimentos}

%A você...\footnote{...principalmente pela atenção até nas notas de rodapé.}



%\end{agradecimentos}
% ---

% ---
% Epígrafe cortazar
% ---
%\begin{epigrafe}
%    \vspace*{\fill}
%	\begin{flushright}
%		\textit{``Quantas vezes me pergunto se isto não é mais do que escrita, numa época em que corremos para o %engano entre equações infalíveis e máquinas de conformismos? Mas perguntar se saberemos encontrar o outro lado do %hábito ou se mais vale se deixar levar pela sua alegre cibernética, não será mais uma vez literatura? Revolta, %conformismo, angústia, alimentos terrestres, todas as dicotomias: o Yin e o Yang, a contemplação (...) e, %finalmente; um encolher de ombros, a paz, o parafuso foi a paz, ninguém podia passar pela rua sem olhar de soslaio %para o parafuso e sentir que ele era a paz. \cite{cortazar1963} }
%	\end{flushright}
%\end{epigrafe}



% ---

% ---
% RESUMOS
% ---

% resumo em português

\setlength{\absparsep}{18pt} % ajusta o espaçamento dos parágrafos do resumo



\begin{resumo}


Esta pesquisa visa problematizar e sistematizar um catálogo de experimentos constituído de pequenas peças musicais e seus algoritmos geradores, objetivando a construção de uma biblioteca de objetos para composição assistida por computador que gere partituras baseadas em regras quantitativas extraídas de análises musicais.

Formalizamos tais aspectos através de um estudo comparado de dois paradigmas de análise musical: \textit{"A Teoria Gerativa da Música Tonal"}\cite{lerdahl1983generative} com algumas de suas continuidades  \cite{lerdahl2009genesis,temperley2001cognition} e a \textit{"Teoria de grupos das classes de alturas"\ (ou "Pitch Class Set Theory")}  \cite{forte1973structure,straus2004}.

Os procedimentos são demonstrados a partir de aspectos singulares de algumas peças da suíte Mikrokosmos do compositor Béla Bartók, gerando composições algorítmicas a partir das regras observadas. Este repertório foi escolhido devido a seu reconhecido contexto como composições pianísticas e pedagógicas situadas nas fronteiras da pós-tonalidade. 

Apontamos as limitações encontradas na aplicação dos paradigmas analíticos adotados aqui no contexto da suíte de peças escolhidas e suas derivações composicionais.

Detalhamos questões computacionais para esta implementação e deixamos um legado de código aberto para continuidades possíveis deste trabalho.


 \textbf{Palavras-chaves}: Música algorítmica. Pós-tonalismo. Teoria dos conjuntos. Pitch class theory. Luteria. Composição assistida por computador. Cibernética. Software livre. Cognição musical. Teoria Gerativa da Música Tonal. Mikrokosmos. Arte Sonora.
\end{resumo}

%%%%%%%%%% traduçoes resumo
\begin{comment}
% resumo em inglês
\begin{resumo}[Abstract]
 \begin{otherlanguage*}{english}
   This is the english abstract.

   \vspace{\onelineskip}
 
   \noindent 
   \textbf{Key-words}: latex. abntex. text editoration.
 \end{otherlanguage*}
\end{resumo}

% resumo em francês 
\begin{resumo}[Résumé]
 \begin{otherlanguage*}{french}
    Il s'agit d'un résumé en français.
 
   \textbf{Mots-clés}: latex. abntex. publication de textes.
 \end{otherlanguage*}
\end{resumo}

% resumo em espanhol
\begin{resumo}[Resumen]
 \begin{otherlanguage*}{spanish}
   Este es el resumen en español.
  
   \textbf{Palabras clave}: latex. abntex. publicación de textos.
 \end{otherlanguage*}
\end{resumo}
% ---
\end{comment}


% ---
% inserir lista de ilustrações
% ---
\pdfbookmark[0]{\listfigurename}{lof}
\listoffigures*
\cleardoublepage
% ---

% ---
% inserir lista de tabelas
% ---
%\pdfbookmark[0]{\listtablename}{lot}
%\listoftables*
%\cleardoublepage
% ---

% ---
% inserir lista de abreviaturas e siglas
% ---
\begin{siglas}
  \item[GTTM] \textit{Generative Theory of Tonal Music}\footnote{ "Teoria Gerativa da Música Tonal"      \cite{lerdahl1983generative} }
  \item[TPS] \textit{Tonal Pitch Space}\footnote{ "Espaço das Alturas Tonais"\cite{lerdahl1988tps} }
  \item[CBMS] \textit{Cognition of Basic Musical Structures}\footnote{ "Cognição das Estruturas Musicais Básicas"\cite{temperley2001cognition} }
  \item[OM] \textit{Open Music}\footnote{ \url{http://repmus.ircam.fr/openmusic/home}. Acessado em 10 de julho de 2014. }
  \item[PD] \textit{Pure Data}\footnote{ \url{http://puredata.info}. Acessado em 10 de julho de 2014. }
\end{siglas}
% ---

% ---
% inserir lista de símbolos
% ---
%\begin{simbolos}
%  \item[$ \Gamma $] Letra grega Gama
%  \item[$ \Lambda $] Lambda
%  \item[$ \zeta $] Letra grega minúscula zeta
%  \item[$ \in $] Pertence
%\end{simbolos}
% ---

% ---
% inserir o sumario
% ---
\pdfbookmark[0]{\contentsname}{toc}
\tableofcontents*
\cleardoublepage
% ---

%
%
%
%
%
%
%
% ----------------------------------------------------------
% ELEMENTOS TEXTUAIS
% ----------------------------------------------------------
\textual

% ----------------------------------------------------------
% Introdução (exemplo de capítulo sem numeração, mas presente no Sumário)
% ----------------------------------------------------------

Este trabalho inicia-se com 

\part{Paradigmas analíticos para um repertorio generativo pos tonal}


\chapter{Percurso pela Analise Musical }




\chapter{Musicologia Assistida por computador}

Prioridade na descrição de métodos da biblioteca music21

Opção por Python http://spectrum.ieee.org/computing/software/top-10-programming-languages 


\section{Analise de Corpus}

\subsection{Music21}

É uma biblioteca projetada para trabalhar com manipulação e análise de \textit{corpus} de arquivos partituráveis\footnote{\url{http://web.mit.edu/music21/doc/moduleReference/moduleCorpus.html} Acesso em 10 de julho de 2014.}. Prepara a conversão entre diversos arquivos de dados musicais (MIDIs, humdrum, lilypond, abc)\footnote{\url{http://web.mit.edu/music21/doc/moduleReference/moduleConverter.html} Acesso em 10 de julho de 2014.}, mas nativamente trabalha com uma estrutura de dados baseada em Music XML.

Music21 tem uma abordagem voltada para uma "musicologia assistida por computador"\ e já tem incorporada em suas classes algumas ferramentas comuns a esta prática como: numeração de grau funcional de acorde\footnote{\url{http://web.mit.edu/music21/doc/moduleReference/moduleRoman.html} Acesso em 10 de julho de 2014.}, numeração de classes de altura usando a classificação de Allen Forte\footnote{\url{http://web.mit.edu/music21/doc/moduleReference/moduleChord.html?\#music21.chord.Chord.forteClassNumber} Acesso em 10 de julho de 2014.} e a implementação dos algoritmos de detecção de tonalidade\footnote{\url{http://web.mit.edu/music21/doc/moduleReference/moduleAnalysisDiscrete.html} Acesso em 10 de julho de 2014.} elaborado por \citeonline{krumhansl1990cognitive} e aperfeiçoado por \citeonline{temperley2001cognition}, descritos nesta pesquisa.\footnote{\autoref{perfiltonal}}

\subsection{Formatos de entrada e saida}

\section{Prolongamentos e inferencia de tonalidade}

\section{Key Profiles}

\section{Escalas e Modalismo}

\section{Contorno melodico}

\section{Metrica composta}

\section{Acento Melodico}

\section{Busca e extraçao de padroes}

Com music21 a possibilidade de tornar a segmentaçao independente do gesto grafico

\subsection{Alternativas em OpenMusic}

as vantagens de segmentaçao via mouse. LZ , Inerface de analise e SOAL como exemplos de possivel automacao do procedimento usando LISP ou servidores externos.

\section{Especialidade da automação versus especialidade do analista}

\chapter{Revisão bibliográfica de estudos Bartokianos}

\subsection{Panorama básico sobre analise bartokiana}

Gillies (05 bartokanalysis pdf) propõe em seu artigo "Bartók Analysis and Authenticity" um panorama dos problemas e lugares comum nas análises de Bartók, apontando alguns critérios para o que poderia ao menos garantir a "autenticidade" entre as diversas correntes analíticas encontradas até então. Gilles inicia a reflexão destacando o notável desafio em argumentarmos qualquer esboço totalizante entre estas composições que sustente a unidade entre os níveis "micro", destacados em apontamentos de interações entre ciclos e grupos intervalares, estratégias modais, polimodais e cromáticas e relações que definam ou sejam definidas pelas "macro" estruturas notáveis em sua obra - como questões sobre o encadeamento de secções por alguma estrutura de prolongamento de expectativa, ambiguidades com alguma sugestão ambígua de tonalidade nos encadeamentos dos grandes blocos, estratégias de simetria ou elaboração de eixos geométricos inspirados na secção áurea ou por vezes o lastro de formas tradicionais como a sonata.

Propõe então a seguinte classificação: análises "autenticas", "semi-autenticas" ou "não-autenticas", sem que nisso haja algum sentido pejorativo, apenas como critério que vai de um historicismo de lastro comprovado até alguma teoria mais inventiva e sem necessidade de comprovação da consciência do compositor sobre estes aspectos, uma teoria comprometida mais com a inspiração de processos criativos derivados. 

A autenticidade seria sobretudo definida pelo registro comprovado de alguma formalização documentada do próprio Bartók, como na compilação "Bela Bartók Essays"(ano). Considera também nesta categoria as pesquisas que a partir dos registros da pesquisa etnomusicológica de Bartók busca fontes originais de estudos dos aspectos folk de seu trabalho. Entram aqui também as analises que tomam em consideração as performances do próprio Bartók ou supervisionadas por ele ainda em vida, para destacar aspectos complementares aos escritos e partituras originais. 

Uma "semi-autenticidade" seria definida a partir de analogias entre influências claras ou declaradas de outros compositores ou contextos de gênero presentes na obra de Bartók, como por exemplo discurso sobre a influência do drama em sua ópera ou a localização de citações paródicas de outros autores em suas peças. Dada autenticidade da analogia portanto, a preocupação fica deslocada para aspectos externos a obra de Bartók.

A "não-autenticidade" comportaria os usos da música de Bartók como exemplo para apontar o funcionamento ou exceção em clichês de harmonia funcional, contraponto, análise shenkeriana ou análise pós-tonal por grupos de classes de alturas. Gilles situa também aqui algumas análises de Bartók que tomam caminhos mais especulativos como as análises de proporção geométrica e simetria propostas por \citeonline{lendvai1971bela}  ou o escrutínio de relações e transformações entre ciclos intervalares, rotações motívicas, coleções modais ou não-diatônicas como no trabalho de \citeonline{antokoletz1984music} 

Em nossa pequena amostra de abordagens sobre alguns traços estruturais na musica de Bartók e seus Mikrokosmos não tem ainda a ambição de esgotar ou mesmo de argumentar uma hierarquia de importâncias destes traços em sua obra como um todo ou na consistência geral de seu estilo. Nossa intenção aqui foi apontar limites e possibilidades para uma automação de manipulação de transformações sugeridas nestas análises e abrir caminho para uma musica generativa inspirada nestes procedimentos.


\subsection{Apontamentos de Lendvai}

a) Afinidades funcionais entre quarto e quinto graus
b) A relação relativa entre grau maior e menor
c) relações de overtone (sobretom)
d)


\cite{lendvai1971bela}





\subsection{As celulas de Alturas - X,Y,Z}

Antokoletz fundamenta boa parte de sua argumentação em seu livro \textit{"The music of Béla Bartók: a study of tonality and progression in twentieth-century music."}\cite{antokoletz1984music} sobre a ideia de subdivisão da oitava em um complexo de ciclos intervalares. Ele insiste por vários ângulos em destacar algumas propriedades da simetria intervalar de sequencias não-diatônicas recorrentes e possibilidade de que houvesse uma estratégia de construção de estruturas transformacionais de grupos de intervalos que chama células X, Y e Z. A nomenclatura "célula de alturas"(\textit{"pitch cell"}) é inspirada nos argumentos sobre construções conjuntos motívicos sobre series baseando-se neste termo proposto por George \citeonline{perle1981serial}


c.f. \cite[p. xiii]{susanni_antokoletz2012music}

Antokoletz localiza com isso também alguns meios de harmonização das melodias modais e polimodais de bases folcloricas onde Bartok buscava estrategias para trabalhar com ambiguidades entre as melodias modais e as tonalidades maiores ou menores associados a centros tonais evidenciados pelas melodias.

With the free use of the folk modes and the subsequent disappear ance of the triad as a basic harmonic premise in the course of Bartók's compositional evolution, the establishment of both local and large-scale
structural coherence became exclusively reliant on intervallic relation-
ships. Bart6k himself commented that the use of the diatonic scale in
the form of the old modes "eventually led to a new conception of the
chromatic scale, every tone of which came to be considered of equal
value and could be used freely and independently.

\subsection{O Eixo de Simetrias}


George Perle alerta para o problema da definição de uma forma de macroestrutura em Bartok não ser suficientemente determinada por estes achados de estratégias internas de construções simétricas.

Impressive as these procedures are, it must be observed that Bartk's symmetrical formations 
are only an incidental aspect of his total compositional means. Even in those few works where they perform a significant structural role they do not ultimately define the context, which is determined
instead by a curious amalgam of various elements..
Can symmetrical formations generate a total musical structure, as triadic relations have done traditionally? The implications of Bartók's work in this, as in other aspects, remain problematical.

Bernard fala em "simetria literal" destacando a observação sobre a conjunção de simetrias intervalares determinadas por posição de registro das alturas de cada um dos componentes de um agregado sonoro (seja uma sequencia de notas em forma melódica ou um cluster vertical). Por exemplo [ Db3, C4, B4 ] possuem entre si as distancias [-11,0,11 ] se considerarmos o C4 como um centro, mas se usarmos {Db4-C4-B4} teremos {-1,0,11} onde apesar de podermos considerar os intervalos [1,11] inversivamente equivalentes por inversão\footnote{Com base nas "teorias de grupos das classes de altura".} desta maneira estes não possuem "literalmente" a mesma distancia.

\citeonline{bernard1986space} localiza no ensaio "Problems of New Music" do proprio Bartók esta ideia onde ele nomeia esta a simetria que considere distâncias por registros de mais de uma oitava de \textit{\textbf{"simetria em espelho"}}


figuras da  \cite[p. 187]{bernard1986space}

\citeonline[p. 189]{bernard1986space} localiza um exemplo aplicado no Concerto n.2 para piano e Orquestra de Bartok uma estrutura de construção de simetrias por alternancia de tons e semitons ao longo de um registro que vai de F4 descendo a C0 nos 5 primeiros compassos e de C5 a Eb0 nos compassos de 6 a 8.

A peça Mikrokosmos n.141 já sugere o procedimento no próprio título "Sujeito e Reflexão".

The piece consists of a series of short sections, each of which is symmetrical about a single
pitch or a pair of pitches one or more octaves apart.\cite[p. 187]{bernard1986space}

Closely related to parallel and mirror symmetry respectively are replica-
tion and inversion. The only difference is that replication and inversion are
better suited to describing order of events, in which a given configuration
may be said to give rise to another.\cite[p. 190]{bernard1986space}


simetria por eixo em sonata para 2 pianos \cite[p. 195-198]{bernard1986space}


\subsection{mikrokosmos 29}

reflexo de imitação - mikro 29

coleção acústica (lidio-mixolidio)

A peça de número 29, chamada de Reflexo de Imitação divide um aspecto
comum a todas as peças iniciais: a composição baseada em pentacordes associados a um
dedilhado que utiliza os cinco dedos de cada mão. Dessa forma, temos na mão direita o
pentacorde diatônico E-Fsus-Gsus-A-B e na mão esquerda, outro pentacorde diatônico, o A-B-C-
D-E. Ambos pertencem a mesma classe de conjuntos 5-27 (02357) e se relacionam por
transposição e inversão (T 8 I). A textura
polifônica
e a imitação das vozes em movimento
Reflexo
em Imitação
Béla Bartók
contrário evidenciam essa relação.

rodrgo-coleção acústica - "compressão intervalar"
comparação entre mkro 29 e 141 ... uso de terminologias de forte
\cite[p. 126]{tymoczko2011geometry}


"simetria inversiva"
"combinação transposicional"
\cite{cohn1988inversional}

One senses in Bart6k's total output an all-encompassing
system of
pitch-relations. The present study is intended to demonstrate that
Bart6k's music is indeed based on such a system.... Pitch relations
in Bart6k's music are primarily
based on the principle of equal subdi-
visions of the octave into the total complex of interval cycles. The fun-
this equal-division
damental concept underlying
system is that of sym-
metry.


Bart6k clearly favored the fifteen inversionally
symmetric tetrachord-classes, in particular those thirteen which
are capable of being realized as symmetric four note pitch-sets.
(The two exceptions are 4-6 [0127] and 4-24 [0248].) The thir-
teen include 4-1 [0123], 4-21 [0246], and 4-9 [0167], which figure
prominently in the writings of Perle and Antokoletz, where
they are called X, Y, and Z cells; 4-17 [0347], Lendvai's
"gamma" chord;11 and 4-3 [0134] and 4-10 [0235], the half-
octatonic tetrachords discussed by Berry.




Example 2a brackets two versions of the four-note motive found in
bb. 1-3 of Bartók's Mikrokosmos, Vol. IV, no. 94, 'From the Island of
Bali'. The first of these motives outlines a descent with the interval
succession 1 / 5/1. This motive - which itself represents a symmetrical
structure in pitch space - forms the basis for a series of motivic trans-
formations that propel the piece forward. In b. 2, the motive turns
upside down, increasing the range of the composition and adding
several new pitches. Taken together, the two pitch collections in bb. 1
and 2 form a one-octave octatonic scale. This symmetrical scale, shown
in Ex. 2b, projects a 1/2 interval series above and below C\#, the axis of
inversion.4 As the piece continues, the motive begins to degenerate
through truncation and reordering, as shown in Ex. 2c. The degenera-
tion of the motive also has the effect of dissolving the symmetry of the
pitch collection. Symmetry is restored in b. 12 where the complete
motive and its inversion return. The piece ends with the symmetrical
collection shown in Ex. 2d. This collection is also drawn from the
octatonic scale, but occupies a broader region in pitch space. Here, the
notes of the opening 1 / 5 / 1 gesture are presented as chords, helping to
highlight the intervallic symmetry of the collection.

 (bartokcrumbmikro.pdf )









Bcla Bartok and
Bulgarian Rhythm", in Bartok Perspectives.
London New York, Oxford Univ.

Press, 2000, pp. 196-212. This distinction by Rice is based on the study Mieczyslaw
Kolinsky: "A Cross
cultural Approach to Mctrorhythmic Patterns", Ethnomusicology
17(1973), pp. 494-506.




\subsection{Apontamentos sobre os Mikrokosmos}

mikrokosmos 25 - exemplo de uso de Armadura de Clave problematizar em AAC. Modalismo ?

Bagatella numero 1 é notada em duas tonalidades diferentes


\cite{suchoff2004bartok}






\part{Formalizaçoes para uma Luteria Composicional sobre regras de estilo}

\chapter{Cliches generativos partituraveis}

\section{Formatos de entrada e saida}

\section{Tecnicas em Music21}

\section{Experimentos em outras linguagens de CAC}

\subsection{Tecnicas em OpenMusic}

\subsection{Problematizacoes em PureData}

\section{Musica e Probabilidade}

\part{Experimentos Generativos}

\chapter{CosmoBagatellas}


\chapter{Lastros e Rumos}







% ----------------------------------------------------------
% ELEMENTOS PÓS-TEXTUAIS
% ----------------------------------------------------------
\postextual
% ----------------------------------------------------------

% ----------------------------------------------------------
% Referências bibliográficas
% ----------------------------------------------------------
%\bibliography{abntex2-modelo-references}
\bibliography{mestrado_glerm}
% ----------------------------------------------------------
% Glossário
% ----------------------------------------------------------
%
% Consulte o manual da classe abntex2 para orientações sobre o glossário.
%
%\glossary

% ----------------------------------------------------------
% Apêndices
% ----------------------------------------------------------

% ---
% Inicia os apêndices
% ---
%\begin{apendicesenv}

% Imprime uma página indicando o início dos apêndices
%\partapendices




%%%%

%\chapter{Repositório de Códigos}
%\label{codigo}


%\subsection{Biblioteca de Algoritmos}



%\lstset{frameround=fttt,language=Python,showspaces=false,
%showtabs=true,tab=\rightarrowfill}
%\begin{lstlisting}[frame=trBL]
%def mod12(n):
%	return n % 12
%
%def note_name(number):
%	notes = "C  D E F G A B".split()
%	return notes[mod12(number)]
%	
%for i in intervalos:
%	if (i in maiores):
%		if (i == (4,7)):
%			tipos.append(("maior",0))
%		if (i == (5,9)):
%			tipos.append(("maior",1))
%		if (i == (3,8)):
%			tipos.append(("maior",2))
% 
%	if (i in menores):
%		if (i == (3,7)):
%			tipos.append(("menor",0))
%		if (i == (5,8)):
%			tipos.append(("menor",1))
%		if (i == (4,9)):
%			tipos.append(("menor",2))
%	if (i in aumentados):
%			tipos.append(("aumentado","not"))
%	if (i in diminutos):
%		if (i == (3,6)):
%			tipos.append(("diminuto",0))
%		if (i == (6,9)):
%			tipos.append(("diminuto",1))
%		if (i == (3,9)):
%			tipos.append(("diminuto",2))
% 
%\end{lstlisting}





%\end{apendicesenv}
% ---


% ----------------------------------------------------------
% Anexos
% ----------------------------------------------------------

% ---
% Inicia os anexos
% ---
%\begin{anexosenv}

% Imprime uma página indicando o início dos anexos
%\partanexos



% ---
%\chapter{Regras da Teoria Gerativa da Musica Tonal}
% ---


%\end{anexosenv}

%---------------------------------------------------------------------
% INDICE REMISSIVO
%---------------------------------------------------------------------
\phantompart
\printindex
%---------------------------------------------------------------------

\end{document}
